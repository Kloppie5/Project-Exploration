\documentclass[11pt, a4paper, draft]{article}
\usepackage[left=0.75in,top=0.6in,right=0.75in,bottom=0.6in]{geometry}
\usepackage[english]{babel}
\usepackage[utf8]{inputenc}

\begin{document}
	
	\section*{ARM mode, Thumb mode, ARM64}
	ARM is a RISC CPU designed with constant-length instructions; ARM mode used instructions encoded in 4 bytes. Because the most common instructions can be encoded using less information, Thumb mode uses instructions encoded in 2 bytes. Thumb-2 is an attempt at extending Thumb mode with a few 4 byte long instructions in order to compete with ARM mode. Thumb-2 is fairly successful, mostly due to XCode (for developing iOS applications) defaults to it.
	
	\section*{Numeral Systems}
	Decimal numbers are represented with a "d" suffix,\\
	Binary numbers are represented with a "0b" prefix or "b" suffix,\\
	Hexadecimal numbers are represented with a "0x" prefix or "h" suffix,\\
	Octal numbers are represented with a "0" prefix
	
	Octal and hexadecimal numbers are usually only used as bit patterns, such as the POSIX file permissions
	
	
	\section*{Empty functions}
	The simplest function is the function that does nothing.
	This is compiled as;
	
		\subsection*{x86}
		ret\\
		this takes the saved address of the callee and jumps to that position
		
		\subsection*{ARM}
		PROC\\
		BX lr\\
		ENDP\\
		ARM does not save the return address on the local stack, but in the link register. BX lr causes execution to jump to that address.
		% lr = link register
		
		\subsection*{MIPS}
		j \$ra\\
		nop\\
		MIPS registers are named by number or by pseudo name.
		% ra = return address
	
	\section*{Returning values}
		\subsection*{x86}
		Data is moved into EAX.
		\subsection*{ARM}
		Data is moved into R0
		\subsection*{MIPS}
		\$V0 (2) register is used to store a return value. LI (Load Immediate) loads the data into the register.
		The returning jump appears before the load immediate, because of a RISC feature called branch delay slot, the instruction following a jump or branch is executed before the jump or branch itself.
	
	\section*{Intel vs AT\&T syntax}
	Intel: instruction destination source
	AT\&T: instruction source destination
	
	Intel: Brackets.
	AT\&T: \% before register names, \$ before numbers. Parentheses.
	
	AT\&T: instruction suffix for operand size;
	q - quad - 64 bit
	l - long - 32 bit
	w - word - 16 bit
	b - byte - 8 bit
	
	64 bit RAX
	last 32 bit EAX
	last 16 bit AX
	split into AH AL, for high and low
	Win64, 4 arguments are passed in RCX, RDX, R8, R9
	Linux, BSD and Mac OSX pass the first 6 arguments in RDI, RSI, RDX, RCX, R8, R9
	
	ARM: 64 bit X- prefix
	32 bit W- prefix
	! after an operand as pre-index
	X29 is FP
	X30 is LR
	
	
	
	MIPS has a "gloabl pointer". Each MIPS instruction has a size of 32 bits, so embedding a 32-bit address is impossible in a single instruction. It is possible to load data from an address 16 bits aways from a register; so register \$gp (28) called the global pointer keeps a 64KiB area of most used data, usually decided by the compiler developers. ELF files have this for the .sbss (uninitialized data) and .sdata (initialized data).
	
\end{document}